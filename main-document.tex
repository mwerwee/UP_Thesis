\documentclass[a4paper,12pt]{report}

% Fill in your details in this file first, used when generating the title page
\def\doctitle{Thesis Title}
\def\authorname{Your full name}
\def\degree{Your Degree}
\def\studentnumber{12345678}

\def\country{Your country}
\def\university{University of Pretoria}
\def\faculty{Your faculty}
\def\department{Your department}
\def\submissiondate{\printdayoff\today}




\usepackage[utf8]{inputenc}

%
\usepackage{titlesec}

% For British English way of writing dates
\usepackage[english]{isodate} 

\usepackage{float}

% For figures
\usepackage{graphicx}

% Set margins to 2 cm
\usepackage[margin=2cm]{geometry}  

% Use the fancy header page style
\usepackage{fancyhdr}  

% For generating the nomenclature
\usepackage{nomencl}

% For generating a glossary for abbreviations, etc.
\usepackage{glossaries} 

% This package ensure that there is no indent when a new paragraph is started
\usepackage[parfill]{parskip}

\makeglossaries
%This is not part of the "front matter", but a document to capture glossary as you go along

% acronyms
\newacronym{gcd}{GCD}{Greatest Common Divisor}

% terms
\newglossaryentry{latex}
{
        name=latex,
        description={Is a mark up language specially suited for 
scientific documents}
}


% If you are using overleaf, you can link your Zotero or Mendeley account to Overleaf.
% Once linked, you will be able to export a bibtex file to your overleaf folder (root in this document) - recompile once you've added a new reference to be able to site it
% To upload from, e.g. Zotero, just click on the "upload" button in overleaf and export your bib file
\usepackage{biblatex} %Imports biblatex package
\addbibresource{references.bib} %Import the bibliography file

\usepackage{appendix}

% For chemistry stuff:
\usepackage[version=4]{mhchem}

% For maths stuff:
\usepackage{amssymb}

% This helps navigation in the document - the red boxes will not be present in your final PDF, but what it boxes in will be a live hyperlink in the document
\usepackage{hyperref}


% The main document

\begin{document}

\pagenumbering{gobble}
\pagenumbering{roman}
\thispagestyle{headings}

\begin{titlepage}
    \begin{center}
        \includegraphics[width=0.6\textwidth]{1-front-matter/up-logo.jpg}
        
        \vspace*{1cm}

        \Large
        \MakeUppercase{\textbf{\doctitle}}
        
        \vspace{0.5cm}
        \large
        by \\
        \vspace{0.5cm}
            
        \authorname
            
        \vfill
            
        A thesis submitted in partial fulfilment of \\
        the requirements for the degree: \\
        
        \vspace{1cm}
        
        Doctor of Philosophy
            
        \vspace{1cm}
            
        in the \\
        \vspace{0.5cm}            
        
        \large
        \faculty \\
        \vspace{0.5cm}
        
        \department \\
        \vspace{0.5cm}

        \university \\
        \country \\
        \vspace{0.5cm}
        \submissiondate
            
    \end{center}
\end{titlepage}
\newpage
\vspace*{\fill}
\section*{\centering Abstract}
\vspace*{\fill}
\newpage
\vspace*{\fill}
\begin{center}
DECLARATION ON PLAGIARISM

UNIVERSITY OF PRETORIA

\end{center}
The University places great emphasis upon integrity and ethical conduct in the preparation of all written work submitted for academic evaluation.
While academic staff teach you about systems of referring and how to avoid plagiarism, you too have a responsibility in this regard. If you are at any stage uncertain as to what is required, you should speak to your lecturer before any written work is submitted. \\ \\
You are guilty of plagiarism if you copy something from a book, article or website without acknowledging the source and pass it off as your own. In effect you are stealing something that belongs to someone else. This is not only the case when you copy work word-by-word (verbatim), but also when you submit someone else's work in a slightly altered form (paraphrase) or use a line of argument without acknowledging it. You are not allowed to use another student's past written work. You are also not allowed to let anybody copy your work with the intention of passing it off as his/her work. \\ \\
Students who commit plagiarism will lose all credits obtained in the plagiarised work. The matter may also be referred to the Disciplinary Committee (Students) for a ruling. Plagiarism is regarded as a serious contravention of the University's rules and can lead to expulsion from the University. \\ \\
The declaration which follows must be appended to all written work submitted within the department. No written work will be accepted unless the declaration has been completed and attached. \\ \\ \\
I, \authorname \\ \\
Student number \studentnumber \\ \\
Topic of work \doctitle \\ \\
Declaration \\
1.	I understand what plagiarism is and am aware of the University's policy in this regard. \\
2.	I declare that this report is my own original work. Where other people's work has been used (from a printed source, internet or any other source), this has been properly acknowledged and referenced in accordance with departmental requirements. \\
3.	I have not used another student's past written work to hand in as my own. \\
4.	I have not allowed, and will not allow, anyone to copy my work with the intention of passing it off as his or her own work. \\ \\ \\ \\
Signature[Add signature image here] \\ \\
\vspace*{\fill}
\newpage
\vspace*{\fill}
\section*{\centering Dedication}
\vspace*{\fill}
\newpage
\vspace*{\fill}
\section*{\centering Acknowledgements}
\vspace*{\fill}

% Table of contents, list of figures, list of tables
\tableofcontents
\listoffigures
\listoftables

% Glossary and nomenclature
% Nomenclature and abbreviations
\clearpage
\section*{Abbreviations}
\begin{table}[H]
	% \centering
	\label{tab:abbreviations}
	\begin{tabular}{p{0.1\linewidth}p{0.8\linewidth}}
		% \hline
		% Symbol & Description  \\ 
		% \hline
		Abbr1 & Abbreviation 1\\
		Abbr2 & Abbreviation 2\\
		% \hline
	\end{tabular} 
\end{table}

\section*{Nomenclature}
\begin{table}[H]
	% \centering
	\label{tab:nomenclature}
 	\begin{tabular}{p{0.1\linewidth}p{0.8\linewidth}}
		% \hline
		% Symbol & Description  \\ 
		% \hline
		$x$ & Some description of x\\
		% \hline
 	\end{tabular} 
\end{table}
\makenomenclature
   

\printglossary
\cleardoublepage

\pagenumbering{arabic}
\pagestyle{headings}
\setlength{\headheight}{15.2pt}
\setlength{\headsep}{10pt}
\pagestyle{fancy}
\renewcommand{\footrulewidth}{0.4pt}

\newpage

\chapter{Chapter title}
\section{Section}
\subsection{A subsection}

This is and example of how the glossary \Gls{latex} and an abbreviation, \gls{gcd} is used.

Here we write some text...

Once a subsection starts, you will notice on the next few pages of the subsection that the headers become a bit fancier.

Lorem ipsum dolor sit amet, consectetur adipiscing elit. Phasellus pretium porta orci, nec sodales nisl consectetur in. In vestibulum nisi convallis mi rutrum, nec sollicitudin neque pharetra. Cras vehicula fermentum massa, convallis mattis libero pellentesque sit amet. Sed cursus mi vitae aliquet vehicula. Nulla a purus sit amet orci pharetra porta. Etiam sem nunc, commodo eu pellentesque ac, iaculis in mi. Cras convallis iaculis sollicitudin. Donec sit amet tincidunt lorem. In sit amet diam viverra, blandit turpis ac, hendrerit urna. Nunc rhoncus diam in neque venenatis pretium. Praesent ligula nibh, laoreet vel felis sed, aliquet porta purus. Pellentesque sed tellus mollis, gravida ante id, mollis urna. Vivamus tempor consequat elit et congue. Donec finibus imperdiet porta. Nunc pellentesque consectetur nibh, in imperdiet nulla rutrum quis.

Some random reference \cite{albertyUseLegendreTransforms2001}

Referencing in text can be done simply, check out examples at:
https://www.imperial.ac.uk/media/imperial-college/administration-and-support-services/library/public/LaTeX-example-Harvard-apr-2019.pdf


\begin{appendices}

\chapter{Some Appendix}
The contents...





\end{appendices}
% heading=bibintoc puts the reference list in the TOC, title=Call the bibliography something else, e.g. References
\printbibliography[heading=bibintoc,title=References]


\end{document}
